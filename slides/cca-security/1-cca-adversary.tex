\begin{frame}{Active adversaries}
What can an \textbf{\textcolor{red}{active adversary}} do?
\begin{itemize}
\item Everything a passive adversary can do (observe)
\item Deviate from the protocol (act)
\item[$\Rightarrow$] Request the decryption of ciphertexts of its choice
\begin{itemize}
\item Ciphertexts may not have been obtained from encryption protocol
\item Malformed ciphertexts could leak information on secret keys!
\end{itemize}
\end{itemize}
\end{frame}

\begin{frame}{Protecting against active adversaries}
\begin{center}
Passive attacks $\Rightarrow$ \textcolor{red}{Confidentiality}
\end{center}
%\vspace{2cm}
\begin{center}
Active attacks $\Rightarrow$ \textcolor{red}{Ciphertext Integrity}

$\rightsquigarrow$ ensures adversary has not deviated from protocol

(no more info than CPA)
\end{center}
\end{frame}

\begin{frame}{CCA security from the Cramer Shoup paradigm [CS02]}
Cramer and Shoup introduce \textbf{Hash Proof Systems} (HPS):
\begin{itemize}
\item[-] Useful tool to abstract the properties of a cryptosystem
\item[-] [CS02] generic construction from HPS to CCA secure encryption
\item[-] Specific properties of HPS ensure \textbf{confidentiality} and \textbf{integrity}
\item[-] [BBL17] generic construction from homomorphic HPS to CCA secure IPFE
\end{itemize}
\end{frame}

\begin{frame}{HPS in the CL framework}
\begin{itemize}
\item We build homomorphic HPS from assumptions in CL framework
\item We improve [BBL17] by tightening security proof
\item We build CCA secure PKE and IPFE from CL framework
\end{itemize}
\end{frame}