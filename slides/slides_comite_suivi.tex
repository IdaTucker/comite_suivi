%\documentclass[10pt,notes]{beamer}       % print frame + notes
%\documentclass[10pt,notes=only]{beamer}   % only notes
\documentclass[10pt]{beamer}  


\usetheme[progressbar=frametitle, numbering=counter]{metropolis}
\usepackage{appendixnumberbeamer}
\usepackage{algcompatible}
\usepackage{algorithm}
\usepackage{algorithmicx}
\usepackage{algpseudocode}
\usepackage{booktabs}
\usepackage{multicol}
\usepackage{eufrak}
\usepackage{tablefootnote}
\usepackage[flushleft]{threeparttable} 
\usepackage[scale=2]{ccicons}
\usetikzlibrary{arrows, calc, positioning, shapes}
\usepackage{tikz-3dplot}
\newenvironment{variableblock}[3]{%
  \setbeamercolor{block body}{#2}
  \setbeamercolor{block title}{#3}
  \begin{block}{#1}}{\end{block}}
\usepackage{pgfplots}
\usepgfplotslibrary{dateplot}
\usepackage[skins,theorems]{tcolorbox}
\usepackage{stackengine}
\usepackage{xspace}
\tcbset{highlight math style={enhanced,
  colframe=mLightBrown,colback=black!2,arc=0pt,boxrule=1pt}}
\usepackage{tabularx} 



\renewcommand\useanchorwidth{T}
\usepackage{xcolor}
\def\theyearwidth{1.5pt}
\newlength\yrsfboxrule
\yrsfboxrule .4\fboxrule

\newcommand\yearwidth[1]{\def\theyearwidth{#1}\ignorespaces}

\newcommand\skipyears[2][white]{%
  \fboxrule\yrsfboxrule%
  \fboxsep=-\yrsfboxrule%
  \fcolorbox{gray}{#1}{\strut\hspace{#2}}%
  \ignorespaces%
}
\newcommand\showyear[2][black]{%
  \fboxsep=0pt%
  \stackon{%
    \colorbox{#1}{\strut\hspace{\theyearwidth}}%
  }{\sffamily\small#2}%
  \ignorespaces%
}

\newcommand{\themename}{\textbf{\textsc{metropolis}}\xspace}

\newcommand\Wider[2][3em]{%
\makebox[\linewidth][c]{%
  \begin{minipage}{\dimexpr\textwidth+#1\relax}
  \raggedright#2
  \end{minipage}%
  }%
}

% Colors
\definecolor{amaranth}{rgb}{0.9, 0.17, 0.31}

% Arrows
\newcommand{\rand}{\xleftarrow{\$}}

% Adversary
\newcommand{\A}{\mathcal{A}}
\newcommand{\B}{\mathcal{B}}
\newcommand{\Adv}{\mathsf{Adv}}

% sets and ensembles
\newcommand{\Z}{\mathbf{Z}}
\newcommand{\N}{\mathbf{N}}
\newcommand{\Zp}{\Z/p\Z}
\newcommand{\Zs}{\Z/s\Z}
\newcommand{\bit}{\{0,1\}}

% PKE
\newcommand{\Kg}{\mathsf{KeyGen}}
% Signatures
\newcommand{\Sign}{\mathsf{Sign}}
\newcommand{\Verif}{\mathsf{Verif}}
\newcommand{\vk}{\mathsf{vk}}

% DDH group with easy DL subgroup generator
\newcommand{\Gen}{\mathsf{Gen}}
\newcommand{\Solve}{\mathsf{Solve}}
% Functional encryption algorithms
\newcommand{\Setup}{\mathsf{Setup}}
\newcommand{\Kd}{\mathsf{KeyDer}}
\newcommand{\Enc}{\mathsf{Enc}}
\newcommand{\Dec}{\mathsf{Dec}}
% Functional encryption master keys
\newcommand{\mpk}{\mathsf{mpk}}
\newcommand{\msk}{\mathsf{msk}}
\newcommand{\pk}{\mathsf{pk}}
\newcommand{\sk}{\mathsf{sk}}
% Functional encryption secret key
\newcommand{\skf}{\mathsf{sk}_F}
\newcommand{\skX}{\mathsf{sk}_X}
\newcommand{\skx}{\mathsf{sk}_{\vec{x}}}
\newcommand{\sol}{\mathsf{sol}}
% Assumptions
\newcommand{\LWE}{\mathsf{LWE}}
\newcommand{\DDH}{\mathsf{DDH}}
\newcommand{\DCR}{\mathsf{DCR}}
\newcommand{\DL}{\mathsf{DL}}
\newcommand{\HSM}{\mathsf{HSM}}
\newcommand{\DDHf}{\mathsf{DDH}\mbox{-}\mathsf{f}}
% Distributions
\newcommand{\D}{\mathcal{D}}

% Defining colors
\definecolor{darkjunglegreen}{rgb}{0.1, 0.6, 0.13} 
\definecolor{themecolor}{rgb}{0.2, 0.7, 0.4} 
\definecolor{yellow}{rgb}{1,0.8,0}
\definecolor{red}{rgb}{0.7,0,0.2}
\definecolor{blue}{rgb}{0.1,0.5,1}
\definecolor{purple}{rgb}{0.5,0.2,0.9}
\definecolor{amethyst}{rgb}{0.6, 0.4, 0.8}
\AtBeginSubsection{\frame{\subsectionpage}}


% Variables in algo
\newcommand{\xbar}{\tilde{\bf{x}}}
\newcommand{\xbari}{{\tilde{\bf{x}}_i}}
\newcommand{\zx}{z_{\textbf{x}}}
\newcommand{\zxi}{z_{\textbf{x}_{i}}}
\newcommand{\cy}{C_{\vec{y}}}


\title{Advanced Cryptographic Systems\\ 
from Homomorphic Encryption modulo \texorpdfstring{$p$}{p}}
\date{}
\author{Ida Tucker\inst{}}
\institute{\inst{}Guilhem Castagnos \& Fabien Laguillaumie
}

\begin{document}

\maketitle

\begin{frame}{Table of contents}
  \setbeamertemplate{section in toc}[sections numbered]
  \tableofcontents[hideallsubsections]
\end{frame}

\section{Linearly Homomorphic Public Key Encryption}

\begin{frame}{Traditional Public Key Encryption (PKE)}
\begin{itemize}
\item \textcolor{mLightBrown}{\textbf{$\Kg$($1^\lambda$)}} $\rightarrow$ secret key $\sk$ and a public key $\pk$
\item \textcolor{mLightBrown}{\textbf{$\Enc$($\pk,m$)}} $\rightarrow$ ciphertext $c$ for $m$ under public key $\pk$
\item \textcolor{mLightBrown}{\textbf{$\Dec$($\sk,c$)}} $\rightarrow$ message $m$
\end{itemize}
Modifying $c \rightsquigarrow c'$ \\
\hspace{1cm} $\Rightarrow$ impossible to decrypt $c'$.
\end{frame}

\begin{frame}{Linearly Homomorphic PKE}
Anyone can publicly modify ciphertexts to manipulate plaintexts
\begin{itemize}
\item Let $\alpha, m, m'\in \mathcal{M}$ and $c \gets \Enc(\pk,m)$, $c'\gets \Enc(\pk, m')$
\item \textcolor{mLightBrown}{\textbf{EvalSum($\pk, c, c'$)}} $\rightarrow$ $c_1$ such that $\Dec(\sk,c_1)=m+m'$
\item \textcolor{mLightBrown}{\textbf{EvalScal($\pk, c, \alpha$)}} $\rightarrow$ $c_2$ such that $\Dec(\sk,c_2)=\alpha m$
\end{itemize}
\end{frame}

\begin{frame}{Security PKE}
\textbf{Security against passive adversaries:}
\begin{itemize}
\item CPA security: Oscar can encrypt plaintexts of his choice \\(Chosen Plaintext Attack)
\end{itemize}
\onslide<2->{\textbf{Security against \textbf{active} adversaries:}
\begin{itemize}
\item CCA security: Oscar wants to decrypt a challenge ciphertext $c$, he can request the decryption of ciphertexts $c'\neq c$ of his choice (Chosen Ciphertext Attack)
\end{itemize}
}
\onslide<3->{\textbf{No CCA security for homomorphic schemes:}
\begin{itemize}
\item Oscar computes $c'\gets \Enc(\pk,0)$ and $c_1\gets EvalSum(\pk, c, c')$
\item Oscar requests the decryption of $c_1$ to get $m$.
\end{itemize}
}
\end{frame}

\begin{frame}{Homomorphic Encryption as Building Block}
\textbf{Paradox of homomorphic encryption:}
\begin{itemize}
\item[-] Cannot achieve maximum security expected of PKE
\onslide<2->{
\item[+] Homomorphic properties allow to build advanced and complex cryptographic systems
}
\end{itemize}
\end{frame}

\section{Linearly Homomorphic PKE from [CL15]}
\subsection{Framework}
\begin{frame}{Framework [CL15]}
The framework of Castagnos and Laguillaumie allows for:
\begin{itemize}
\item Linearly Homomorphic Encryption Schemes
\begin{itemize}
\item \textbf{Message Space of Prime order}
\item Efficient decryption
\end{itemize}
\item Well studied mathematical assumptions $\Rightarrow$ Provable Security
\item Concrete instantiation (in class groups)
\end{itemize}
\end{frame}

\begin{frame}{Framework (sketch) \cite{RSA:CasLag15}}
\begin{block}{Group with an easy discrete logarithm ($\DL$) subgroup}

\begin{itemize}
\item \textcolor{mLightBrown}{$G=\langle g\rangle$} cyclic group of order $p\cdot s$ such that \textcolor{mLightBrown}{$\gcd(p,s)=1$}.
\item \textcolor{mLightBrown}{$p$} large prime
\item \textcolor{mLightBrown}{$F=\langle f \rangle$} subgroup of $G$ 
of order $p$.
\item \textcolor{mLightBrown}{$G^p= \langle g_p\rangle$} $=\{x^p, x \in G\}$ subgroup of $G$
of order $s$,$$G = F \times G^p.$$
\vspace{-0.3cm}
\item \textcolor{mLightBrown}{$\DL$} is \textbf{easy} in $F$\hspace{1cm} ($\DL$: given $f$ and $h=f^{\textcolor{amaranth}{x}}$, find $\textcolor{amaranth}{x}\in\Zp$)
\end{itemize}
%   
\end{block}
\end{frame}
\note{Framework \cite{RSA:CasLag15}.
We use the framework introduced by Castagnos and Laguillaumie at CTRSA 2015 which supposes the existence of a DDH group with an easy DL subgroup: a cyclic group $G = \langle g \rangle$ where the DDH assumption holds together with a subgroup $F = \langle f \rangle$ of G where the discrete logarithm problem is easy.}

{\setbeamercolor{palette primary}{fg=black!2, bg=mDarkTeal}
\begin{frame}[standout]{New Assumption}
Hard Subgroup Membership problem \textcolor{mLightBrown}{$\HSM$}: \\Hard to distinguish $p$-th powers in $G$ \\$\{x\rand G\} \approx_c \{x\rand G^p\}$.
\end{frame}
}

\subsection{The Linearly Homomorphic PKE Scheme}

\begin{frame}{Homomorphic PKE scheme mod $p$ from $\HSM$}
\begin{figure}[H]
\begin{tikzpicture}

% Setup x,y coordinates; width and height
\def\xsetup{-1}
\def\ysetup{9}
\def\wsetup{11}
\def\hsetup{1.2}
\def\topsetup{\ysetup+\hsetup}

% Enc x,y coordinates; width and height
\def\wenc{11}
\def\henc{2.1}
\def\xenc{-1}
\def\yenc{\ysetup - \henc - 0.2}
\def\topenc{\yenc+\henc}

% Decryption x,y coordinates; width and height
\def\hdec{1.7}
\def\wdec{11}
\def\xdec{-1}
\def\ydec{\yenc-\hdec-0.2}
\def\topdec{\ydec+\hdec}

% Normal spacing top and first text
\def\ntopspace{0.3}

% Orange boxes:
\onslide<1>{
% Setup 
 \draw[rounded corners=5pt, mLightBrown, fill=mLightBrown]
  (\xsetup,\ysetup) rectangle ++(\wsetup,\hsetup);
}

\onslide<2>{
% Encryption
 \draw[rounded corners=5pt, mLightBrown, fill=mLightBrown]
  (\xenc,\yenc) rectangle ++(\wenc,\henc);
}


% Setup
\onslide<1->{
\node[align=left] at  (\xsetup + 0.7 ,\topsetup - \ntopspace) {\underline{$\Kg$}};

\node[align=left] at  (\xsetup + 3 ,\topsetup - \ntopspace) 
	{Sample  ${t}$}; %    \hookleftarrow \D_p$};\\
\node[align=left] at  (\xsetup + 6.2 ,\topsetup - \ntopspace)  
	{and compute $\quad h = g_p^{t}$};
\node[align=left] at  (\xsetup + \wsetup/2 ,\topsetup - \ntopspace - 0.5)  
	{$\sk = {t}\;$ and $\;\pk = h$}; 
}

\onslide<2->{
% Setup 
 \draw[rounded corners=5pt]
  (\xsetup,\ysetup) rectangle ++(\wsetup,\hsetup);
% Encryption
\node[align=left] at  (\xenc + 0.45,\topenc - 0.25) {\underline{\textbf{Enc}}};
\node[align=left] at  (\xenc + 3.7,\topenc - 1) 
	{
	\textbf{Plaintext:} $m \in \Zp$\\
	\textbf{Sample randomness} $r$\\%    \hookleftarrow \D_p$\\
    \textbf{Ciphertext:}\\
    \hspace{2cm}$(C_0, C_1) = (g_p^r, f^{m}\cdot h^r)$};
}


\onslide<3->{
% Encryption
 \draw[rounded corners=5pt]
  (\xenc,\yenc) rectangle ++(\wenc,\henc);
% Decryption
 \draw[rounded corners=5pt, mLightBrown, fill=mLightBrown]
  (\xdec,\ydec) rectangle ++(\wdec,\hdec);

\node[align=left] at  (\xdec + 0.45 ,\topdec - 0.3) {\underline{\textbf{Dec}}};
\node[align=left] (input) at (\xdec + 3,\topdec - 0.3) 
	{\textbf{From $(C_0,C_1)$ and $\sk = t$ :}};

\node[align=left] at (\xdec + 4, \topdec- 1) {$C_1/C_0^{t}$};
\draw[-triangle 60, mDarkBrown] (\xdec + 5.3, \topdec-1) -- ++(1,0) 
	node [midway,above] {$\DL$};
            \node[align=left] at (\xdec + 7.5, \topdec -1) 
            {$m \mod p$};

}
\end{tikzpicture}
\end{figure}
\end{frame}



{\setbeamercolor{palette primary}{fg=black!2, bg=mDarkTeal}
\begin{frame}[standout]{Security}
  This scheme is \textcolor{mLightBrown}{CPA secure} under the \textcolor{mLightBrown}{$\HSM$} assumption.
\end{frame}
}

%\begin{frame}{Game 0: the original security experiment}
\begin{figure}
\begin{tikzpicture}

\def\xAdv{0}
\def\yAdv{1.5}
\def\wAdv{1.5}
\def\hAdv{6}
\def\topAdv{\yAdv+\hAdv}

\def\xChal{3}
\def\yChal{1.5}
\def\wChal{5.8}
\def\hChal{6}
\def\topChal{\yChal+\hChal}

% Adversary box
\draw[rounded corners=10pt]
  (\xAdv,\yAdv) rectangle ++(\wAdv,\hAdv);

\node[align=left] at (\xAdv + 0.5,\topAdv - 0.3) {\textsf{Adv}};

%%%%%%%%%%%%%%%%%%
% Challenger box %
%%%%%%%%%%%%%%%%%%

\draw[rounded corners=10pt, fill=mDarkTeal, mDarkTeal]
  (\xChal, \yChal) rectangle ++(\wChal, \hChal);
\node[align=left] at (\xChal + 1,\topChal - 0.3)
 	{\textbf{\textcolor{mLightBrown}{\textsf{Challenger}}}};

% Setup
\node[align=left] (setup) at (\xChal + 5,\topChal -1.25)
	{\textbf{\textcolor{black!2}{$\Setup$}}};
 
\draw[-triangle 60, black!2] (setup) -- ++(-1.5,0);
 
\node[align=left] at (\xChal + 1.5,\topChal -1) 
  	{\textbf{\textcolor{black!2}{$\sk = t \hookleftarrow \D_p$}}}; 
\node[align=left] at (\xChal + 1.5,\topChal -1.5)
 	{\textbf{\textcolor{amaranth}{$\pk = h = g_p^t$}}}; 

% Sample bit
\node[align=left] at (\xChal + 1.45,\topChal - 3)
 	{\textbf{\textcolor{black!2}{$b^* \rand \bit$}}}; 
    
% Encryption
\node[align=left] at (\xChal + 1.2,\topChal - 3.6)
 	{\textbf{\textcolor{black!2}{$r\hookleftarrow \mathcal{D}_p$}}}; 

\node[align=left] at (\xChal + 2.6,\topChal - 4.6)
 	{\textcolor{amaranth}{$(C_0, C_1)$}\textcolor{black!2}{$= (g_p^r, C_1=f^{m_{b^*}}\cdot h^r)$}};

% Output
\node[align=left] at (\xChal + \wChal/2,\yChal + 0.3)
 	{\textbf{\textcolor{black!2}{Output $(b=b^*)$}}}; 

% Arrows between Challenger and Adv
\draw[-triangle 60, amaranth] (\xChal,\topChal - 1.25) -- ++(-1.5,0) 
	node [midway,above] {$\pk$};
   
\draw[-triangle 60, mDarkTeal] (\xChal-1.5,\topChal - 3) -- ++(1.5,0) 
	node [midway,above] {$m_0, m_1$};
   
\draw[-triangle 60, amaranth] (\xChal,\topChal - 4.6) -- ++(-1.5,0) 
	node [midway,above] {$(C_0, C_1)$};
   
\draw[-triangle 60, mDarkTeal] (\xChal-1.5,\topChal - 5.5) -- ++(1.5,0) 
	node [midway,above] {$b$};
   

\end{tikzpicture}
\end{figure}
\begin{center}
\large Game 0 is the original security experiment.
\end{center} 
\end{frame}

%\input{hsm-pke/04-pke-game1.tex}

%\begin{frame}{Game 2: use $\sk$ to compute $(C_0,C_1)$}
\begin{figure}
\begin{tikzpicture}

\def\xAdv{0}
\def\yAdv{1.5}
\def\wAdv{1.5}
\def\hAdv{6}
\def\topAdv{\yAdv+\hAdv}

\def\xChal{3}
\def\yChal{1.5}
\def\wChal{5.8}
\def\hChal{6}
\def\topChal{\yChal+\hChal}

% Adversary box
\draw[rounded corners=10pt]
  (\xAdv,\yAdv) rectangle ++(\wAdv,\hAdv);

\node[align=left] at (\xAdv + 0.5,\topAdv - 0.3) {\textsf{Adv}};

%%%%%%%%%%%%%%%%%%
% Challenger box %
%%%%%%%%%%%%%%%%%%

\draw[rounded corners=10pt, fill=mDarkTeal, mDarkTeal]
  (\xChal, \yChal) rectangle ++(\wChal, \hChal);
\node[align=left] at (\xChal + 1,\topChal - 0.3)
 	{\textbf{\textcolor{mLightBrown}{\textsf{Challenger}}}};

 
\node[align=left] at (\xChal + 1.5,\topChal -1) 
  	{\textbf{\textcolor{black!2}{$\sk = t \hookleftarrow$}\textcolor{black!2}{$\D$}}}; 
\node[align=left] at (\xChal + 1.5,\topChal -1.5)
 	{\textbf{\textcolor{amaranth}{$\pk = h = g_p^t$}}}; 

% Sample bit
\node[align=left] at (\xChal + 1.45,\topChal - 3)
 	{\textbf{\textcolor{black!2}{$b^* \rand \bit$}}}; 
    
% Encryption
\node[align=left] at (\xChal + 1.2,\topChal - 3.6)
 	{\textbf{\textcolor{black!2}{$r\hookleftarrow \mathcal{D}_p$}}}; 

\node[align=left] at (\xChal + 2.6,\topChal - 4.6)
 	{\textcolor{amaranth}{$(C_0, C_1)$}\textcolor{black!2}{$= (g_p^r, $}  
	\textcolor{mLightBrown}{$f^{m_{b^*}}\cdot C_0^{t}$}
	\textcolor{black!2}{$)$}
};

% Output
\node[align=left] at (\xChal + \wChal/2,\yChal + 0.3)
 	{\textbf{\textcolor{black!2}{Output $(b=b^*)$}}}; 

% Arrows between Challenger and Adv
\draw[-triangle 60, amaranth] (\xChal,\topChal - 1.25) -- ++(-1.5,0) 
	node [midway,above] {$\pk$};
   
\draw[-triangle 60, mDarkTeal] (\xChal-1.5,\topChal - 3) -- ++(1.5,0) 
	node [midway,above] {$m_0, m_1$};
   
\draw[-triangle 60, amaranth] (\xChal,\topChal - 4.6) -- ++(-1.5,0) 
	node [midway,above] {$(C_0, C_1)$};
   
\draw[-triangle 60, mDarkTeal] (\xChal-1.5,\topChal - 5.5) -- ++(1.5,0) 
	node [midway,above] {$b$};
   

\end{tikzpicture}
\end{figure}
\begin{center}
\large From $\A$'s view, Games $1$ and $2$ are identical.
\end{center} 
\end{frame}


%\begin{frame}{Game 3: compute $C_0\in G\backslash G^p$}
\begin{figure}
\begin{tikzpicture}

\def\xAdv{0}
\def\yAdv{1.5}
\def\wAdv{1.5}
\def\hAdv{6}
\def\topAdv{\yAdv+\hAdv}

\def\xChal{3}
\def\yChal{1.5}
\def\wChal{5.8}
\def\hChal{6}
\def\topChal{\yChal+\hChal}

% Adversary box
\draw[rounded corners=10pt]
  (\xAdv,\yAdv) rectangle ++(\wAdv,\hAdv);

\node[align=left] at (\xAdv + 0.5,\topAdv - 0.3) {\textsf{Adv}};

%%%%%%%%%%%%%%%%%%
% Challenger box %
%%%%%%%%%%%%%%%%%%

\draw[rounded corners=10pt, fill=mDarkTeal, mDarkTeal]
  (\xChal, \yChal) rectangle ++(\wChal, \hChal);
\node[align=left] at (\xChal + 1,\topChal - 0.3)
 	{\textbf{\textcolor{mLightBrown}{\textsf{Challenger}}}};

 
\node[align=left] at (\xChal + 1.5,\topChal -1) 
  	{\textbf{\textcolor{black!2}{$\sk = t \hookleftarrow \D$}}}; 
\node[align=left] at (\xChal + 1.5,\topChal -1.5)
 	{\textbf{\textcolor{amaranth}{$\pk = h = g_p^t$}}}; 

\onslide<2->{
\draw[-, amaranth] (\xChal + 2.2,\topChal -1.8) -- ++(0,-0.3);
\draw[-triangle 60, amaranth] (\xChal + 2.2,\topChal -2.1) -- ++(1,0) 
	node [right, amaranth] {fixes $t\mod s$};
}

% Sample bit
\node[align=left] at (\xChal + 1.45,\topChal - 2.9)
 	{\textbf{\textcolor{black!2}{$b^* \rand \bit$}}}; 
    
% Encryption
\node[align=left] at (\xChal + 1.2,\topChal - 3.5)
 	{\textbf{\textcolor{black!2}{$r\hookleftarrow \mathcal{D}_p$}}}; 
\node[align=left] at (\xChal + 3,\topChal - 3.5)
 	{\textbf{\textcolor{mLightBrown}{ and $u \hookleftarrow \Zp$}}}; 

\onslide<1-3>{
\node[align=left] at (\xChal + 2.6,\topChal - 4.6)
 	{\textcolor{amaranth}{$(C_0, C_1)$} \textcolor{black!2}{$= ( $} \textcolor{mLightBrown}{$g_p^r\cdot f^u, $} \textcolor{black!2}{$f^{m_{b^*}}\cdot C_0^t)$}};
}
\onslide<3->{
\draw[-, amaranth] (\xChal + 0.7,\topChal -4.3) -- ++(0,0.2);
\draw[-triangle 60, amaranth] (\xChal + 0.7,\topChal -4.1) -- ++(0.5,0) 
	node [right, amaranth] {fixes $r\mod s$ and $u\mod p$};
}

\onslide<4->{
\node[align=left] at (\xChal + 2.8,\topChal - 4.6)
 	{\textcolor{amaranth}{$(C_0, C_1)$}\textcolor{black!2}{$= ($} \textcolor{mLightBrown}{$g_p^r\cdot f^u, $} \textcolor{black!2}{$f^{m_{b^*}+u\cdot t}\cdot h^{r})$}};
	
\draw[-, amaranth] (\xChal + 1.2,\topChal -4.9) -- ++(0,-0.2);
\draw[-triangle 60, amaranth] (\xChal + 1.2,\topChal -5.1) -- ++(0.4,0) 
	node [right, amaranth] {reveals $m_{b^*} + u \cdot t\mod p$};
}


% Output
\node[align=left] at (\xChal + \wChal/2,\yChal + 0.3)
 	{\textbf{\textcolor{black!2}{Output $(b=b^*)$}}}; 

% Arrows between Challenger and Adv
\draw[-triangle 60, amaranth] (\xChal,\topChal - 1.25) -- ++(-1.5,0) 
	node [midway,above] {$\pk$};
   
\draw[-triangle 60, mDarkTeal] (\xChal-1.5,\topChal - 3) -- ++(1.5,0) 
	node [midway,above] {$m_0, m_1$};
   
\draw[-triangle 60, amaranth] (\xChal,\topChal - 4.6) -- ++(-1.5,0) 
	node [midway,above] {$(C_0, C_1)$};
   
\draw[-triangle 60, mDarkTeal] (\xChal-1.5,\topChal - 5.5) -- ++(1.5,0) 
	node [midway,above] {$b$};
   
\end{tikzpicture}
\end{figure}
\begin{center}
\large Games $2$ and $3$ are undistinguishable to $\A$ \\under the \textcolor{mLightBrown}{$\HSM$ assumption}.
\end{center} 
\end{frame}



%%%%%%%%%%%%%%%%%%%%%%%%%%%%%%%%%%%%%%%%%%%%%%%%%%%%%%%%%%%%%%%%%%%%%%%%%%%%%%%
\section{Inner Product Functional Encryption (IPFE)}
\subsection{Functional Encryption (FE)}
\begin{frame}{Traditional Encryption: All or Nothing}
\begin{figure}
    \centering
\begin{tikzpicture}
    % Variables
% Context x,y coordinates; width and height
\def\xalice{0}
\def\yalice{7}
\def\walice{1.5}
\def\halice{0.6}
\def\topalice{\yalice+\halice}

% bob x,y coordinates; width and height
\def\xbob{6}
\def\ybob{7}
\def\wbob{1.5}
\def\hbob{.6}
\def\topbob{\ybob+\hbob}

 \draw[rounded corners=5pt, mLightBrown, fill=mLightBrown]
  (\xbob,\ybob) rectangle ++(\wbob,\hbob);
  \node[align=left] at  (\xbob + \wbob/2 ,\ybob+\hbob/2) 
	{Bob};
\node[align=left, mLightBrown] at  (\xbob + \wbob/2 ,\ybob-\hbob/2) 
	{$(\pk_{Bob}, \sk_{Bob})$};
  
 \draw[rounded corners=5pt]
  (\xalice,\yalice) rectangle ++(\walice,\halice);
\node[align=left] at  (\xalice + \walice/2 ,\yalice+\halice/2) 
	{Alice};
\node[align=left] at  (\xalice + \walice/2 ,\yalice-\halice/2) 
	{m};	
	
{
\draw[-triangle 60, mLightBrown] (\xbob-0.8,\ybob-1) -- ++(-2.7,0) 
	node [midway,above] {$\pk_{Bob}$};
}

{
\node[align=left] at  (\xalice + \walice/2 ,\yalice-1.5) 
	{$C =\Enc (\pk_{Bob}, m)$};
\draw[-triangle 60] (\xalice + \walice + 1,\yalice-2) -- ++(2.7,0) 
	node [midway,above] {$C$};
}

{
\node[align=left, mLightBrown] at  (\xbob + \wbob/2 ,\ybob-2.5) 
	{$m=\Dec(\sk_{Bob},C)$};
}
{
\node[align=left] at  (3.5 ,\ybob-3.5) 
	{\large Bob gets \textbf{all} the information in $m$.};
}
\end{tikzpicture}
\end{figure}
\end{frame}

\begin{frame}{Fine Grained Access to Info with Traditional Encryption}
\begin{figure}
    \centering
\begin{tikzpicture}
    % Variables
% Alice x,y coordinates; width and height
\def\xEmployees{8}

\def\yEmployeeProfile{8}
\def\yOfficeWorker{6}
\def\yTelephoneOperator{4}
\def\yManWithSunglasses{2}


% Encryptor x,y coordinates; width and height
\def\xEncryptor{0}
\def\yEncryptor{5}

% Other
\def\xMiddle{4}


\node[inner sep=0pt] at (\xEmployees ,\yEmployeeProfile)
 {\includegraphics[width=0.1\textwidth]{img/employee-profile.png}};
 
\node[inner sep=0pt] at (\xEmployees ,\yOfficeWorker)
 {\includegraphics[width=0.1\textwidth]{img/office-worker-outline.png}};

\node[inner sep=0pt] at (\xEmployees ,\yTelephoneOperator)
 {\includegraphics[width=0.1\textwidth]{img/telephone-operator.png}};
 
\node[inner sep=0pt] at (\xEmployees ,\yManWithSunglasses)
 {\includegraphics[width=0.1\textwidth]{img/man-with-sunglasses.png}};

\node[inner sep=0pt] at (\xEncryptor ,\yEncryptor)
 {\includegraphics[width=0.1\textwidth]{img/document.png}};

\onslide<1,2,3>
{
\node[align=center, yellow] at (\xEmployees ,\yEmployeeProfile-0.8) {$\pk_1,\sk_1$}; 
\node[align=center, blue] at (\xEmployees ,\yOfficeWorker-0.8) {$\pk_2,\sk_2$}; 
\node[align=center, purple] at (\xEmployees ,\yTelephoneOperator-0.8) {$\pk_3,\sk_3$}; 
\node[align=center, amaranth] at (\xEmployees ,\yManWithSunglasses-0.8) {$\pk_4,\sk_4$}; 
\node[align=center, black] at (\xEncryptor ,\yEncryptor-0.8) {$m$}; 
}

\onslide<2>{
\draw[-triangle 60, yellow, thick] (\xEncryptor ,\yEncryptor-1) to (\xEncryptor-1.5, \yEncryptor-2);
\draw[-triangle 60, blue, thick] (\xEncryptor ,\yEncryptor-1) to (\xEncryptor-0.5, \yEncryptor-2);
\draw[-triangle 60, purple, thick] (\xEncryptor ,\yEncryptor-1) to (\xEncryptor+0.5, \yEncryptor-2);
\draw[-triangle 60, amaranth, thick] (\xEncryptor ,\yEncryptor-1) to (\xEncryptor+1.5, \yEncryptor-2);

\node[align=center, yellow] at (\xEncryptor-1.5, \yEncryptor-2.4) {$m_1$}; 
\node[align=center, blue] at (\xEncryptor-0.5, \yEncryptor-2.4) {$m_2$}; 
\node[align=center, purple] at (\xEncryptor+0.5, \yEncryptor-2.4) {$m_3$}; 
\node[align=center, amaranth] at (\xEncryptor+1.5, \yEncryptor-2.4) {$m_4$}; 
}

\onslide<3>
{
\draw[-triangle 60, yellow, thick] (\xEncryptor+1, \yEncryptor) to (\xMiddle, \yEmployeeProfile-0.2);
\draw[-triangle 60, blue, thick] (\xEncryptor+1, \yEncryptor) to (\xMiddle,\yOfficeWorker);
\draw[-triangle 60, purple, thick] (\xEncryptor+1, \yEncryptor) to (\xMiddle,\yTelephoneOperator);
\draw[-triangle 60, amaranth, thick] (\xEncryptor+1, \yEncryptor) to (\xMiddle,\yManWithSunglasses);

\node[align=center, yellow] at (\xEncryptor ,\yEncryptor-1.2) {$C_1 =\Enc (\pk_1, m_1)$}; 
\node[align=center, blue] at (\xEncryptor ,\yEncryptor-1.6) {$C_2 =\Enc (\pk_{2}, m_2)$}; 
\node[align=center, purple] at (\xEncryptor ,\yEncryptor-2) {$C_3 =\Enc (\pk_{3}, m_3)$}; 
\node[align=center, amaranth] at (\xEncryptor ,\yEncryptor-2.4) {$C_4 =\Enc (\pk_{4}, m_4)$}; 
\node[align=center, amaranth] at (\xEncryptor ,\yEncryptor-2.8) {$\textcolor{blue}{C_5} =\Enc (\pk_{4}, \textcolor{blue}{m_2})$}; 


\node[align=center,yellow] at (\xMiddle +1 ,\yEmployeeProfile-0.2) {$C_1$}; 
\node[align=center,blue] at (\xMiddle +1,\yOfficeWorker) {$C_2 $}; 
\node[align=center,purple] at (\xMiddle +1,\yTelephoneOperator) {$C_3$}; 
\node[align=center, amaranth] at (\xMiddle+1 ,\yManWithSunglasses) {$C_4,\textcolor{blue}{C_5}$}; 
}
\end{tikzpicture}
\end{figure}

\end{frame}


\begin{frame}{Ideal Fine Grained Access to Information}
\begin{figure}
    \centering
\begin{tikzpicture}
    % Variables
% Alice x,y coordinates; width and height
\def\xEmployees{8}

\def\yEmployeeProfile{8}
\def\yOfficeWorker{6}
\def\yTelephoneOperator{4}
\def\yManWithSunglasses{2}


% Encryptor x,y coordinates; width and height
\def\xEncryptor{0}
\def\yEncryptor{5}

% Other
\def\xMiddle{4}


\node[inner sep=0pt] at (\xEmployees ,\yEmployeeProfile)
 {\includegraphics[width=0.1\textwidth]{img/employee-profile.png}};
 
\node[inner sep=0pt] at (\xEmployees ,\yOfficeWorker)
 {\includegraphics[width=0.1\textwidth]{img/office-worker-outline.png}};

\node[inner sep=0pt] at (\xEmployees ,\yTelephoneOperator)
 {\includegraphics[width=0.1\textwidth]{img/telephone-operator.png}};
 
\node[inner sep=0pt] at (\xEmployees ,\yManWithSunglasses)
 {\includegraphics[width=0.1\textwidth]{img/man-with-sunglasses.png}};

\node[inner sep=0pt] at (\xEncryptor ,\yEncryptor)
 {\includegraphics[width=0.1\textwidth]{img/document.png}};


\node[align=center, yellow] at (\xEmployees ,\yEmployeeProfile-0.8) {$\sk_1$}; 
\node[align=center, blue] at (\xEmployees ,\yOfficeWorker-0.8) {$\sk_2$}; 
\node[align=center, purple] at (\xEmployees ,\yTelephoneOperator-0.8) {$\sk_3$}; 
\node[align=center, amaranth] at (\xEmployees ,\yManWithSunglasses-0.8) {$\sk_4$, \textcolor{blue}{$\sk_2$}}; 
\node[align=center, black] at (\xEncryptor ,\yEncryptor-0.8) {$\pk, m$}; 


\draw[-triangle 60, black, thick] (\xEncryptor+1, \yEncryptor) to (\xMiddle, \yEmployeeProfile-0.2);
\draw[-triangle 60, black, thick] (\xEncryptor+1, \yEncryptor) to (\xMiddle,\yOfficeWorker);
\draw[-triangle 60, black, thick] (\xEncryptor+1, \yEncryptor) to (\xMiddle,\yTelephoneOperator);
\draw[-triangle 60, black, thick] (\xEncryptor+1, \yEncryptor) to (\xMiddle,\yManWithSunglasses);

\node[align=center, black] at (\xEncryptor ,\yEncryptor-1.2) {$C =\Enc (\pk, m)$}; 

\node[align=center,yellow] at (\xMiddle +2 ,\yEmployeeProfile-0.2) {$m_1=\Dec(\sk_1, C)$}; 
\node[align=center,blue] at (\xMiddle +2,\yOfficeWorker) {$m_2=\Dec(\sk_2, C)$}; 
\node[align=center,purple] at (\xMiddle +2,\yTelephoneOperator) {$m_3=\Dec(\sk_3, C)$}; 
\node[align=center, amaranth] at (\xMiddle+2 ,\yManWithSunglasses) {$m_4=\Dec(\sk_4, C)$}; 
\node[align=center,blue] at (\xMiddle +2,\yManWithSunglasses-0.4) {$m_2=\Dec(\sk_2, C)$}; 

\end{tikzpicture}
\end{figure}

\end{frame}


\note{Suppose we want a more fine grained access to info, in a company, $\neq$ employees may be granted $\neq$ access to info, depending on how long they have been working for the company, etc. 

So for a given file $m$, $\neq$ employees will have access to $\neq$ info in $m$.

Say $m_1,m_2,m_3$ and $m_4$ are different parts of the info contained in $m$, and to which the yellow, blue, purple and red employees are respectively granted access. So the yellow employee is allowed to know $m_1$, the blue employee is allowed to know $m_2$ etc.

With traditional enc $\Rightarrow$ each employee must have his own pub and priv key pair,
\textbf{(click)}
 and the specific piece of information that a user is granted access to will be encrypted under his public key.
 
 But clearly this yields a lot of encryption keys, a lot of $\neq$ ciphertexts $\Rightarrow$ messy and heavyweight.
 Since all the info in $m_1,m_2,m_3$ and $m_4$ is actually contained in $m$,
 what we would like \textbf{(click)} is to have a single public key with which we encrypt the ciphertext, and thus a single ciphertext. But each employee possesses a $\neq$ decryption key which is associated to the part of info he is allowed to get about $m$.
 
 And so everyone gets the same ciphertext $C$, but depending on the secret key an employee owns, decryption recovers a different part of the information in $m$.

}


\begin{frame}{Functional Encryption}
\begin{figure}
    \centering
\begin{tikzpicture}
    % Variables
% Alice x,y coordinates; width and height
\def\xEmployees{8}

\def\yEmployeeProfile{8}
\def\yOfficeWorker{6}
\def\yTelephoneOperator{4}
\def\yManWithSunglasses{2}


% Encryptor x,y coordinates; width and height
\def\xEncryptor{0}
\def\yEncryptor{5}

% Other
\def\xMiddle{4}


\node[inner sep=0pt] at (\xEmployees ,\yEmployeeProfile)
 {\includegraphics[width=0.1\textwidth]{img/employee-profile.png}};
 
\node[inner sep=0pt] at (\xEmployees ,\yOfficeWorker)
 {\includegraphics[width=0.1\textwidth]{img/office-worker-outline.png}};

\node[inner sep=0pt] at (\xEmployees ,\yTelephoneOperator)
 {\includegraphics[width=0.1\textwidth]{img/telephone-operator.png}};
 
\node[inner sep=0pt] at (\xEmployees ,\yManWithSunglasses)
 {\includegraphics[width=0.1\textwidth]{img/man-with-sunglasses.png}};

\node[inner sep=0pt] at (\xEncryptor ,\yEncryptor)
 {\includegraphics[width=0.1\textwidth]{img/document.png}};

\node[align=center, yellow] at (\xEmployees ,\yEmployeeProfile-0.8) {$\sk_{F_1}$}; 
\node[align=center, blue] at (\xEmployees ,\yOfficeWorker-0.8) {$\sk_{F_2}$}; 
\node[align=center, purple] at (\xEmployees ,\yTelephoneOperator-0.8) {$\sk_{F_3}$}; 
\node[align=center, amaranth] at (\xEmployees ,\yManWithSunglasses-0.8) {$\sk_{F_4}, \textcolor{blue}{\sk_{F_2}}$}; 
%\node[align=center, black] at (\xEncryptor ,\yEncryptor+1) {$\msk, \pk$}; 
\node[align=center, black] at (\xEncryptor ,\yEncryptor-0.8) {$m$}; 

\onslide<1>{
\draw[-triangle 60, yellow, thick] (\xEncryptor ,\yEncryptor-1) to (\xEncryptor-1.5, \yEncryptor-2);
\draw[-triangle 60, blue, thick] (\xEncryptor ,\yEncryptor-1) to (\xEncryptor-0.5, \yEncryptor-2.7);
\draw[-triangle 60, purple, thick] (\xEncryptor ,\yEncryptor-1) to (\xEncryptor+0.5, \yEncryptor-2.7);
\draw[-triangle 60, amaranth, thick] (\xEncryptor ,\yEncryptor-1) to (\xEncryptor+1.5, \yEncryptor-2);

\node[align=center, yellow] at (\xEncryptor-1.7, \yEncryptor-2.3) {$m_1=F_1(m)$}; 
\node[align=center, blue] at (\xEncryptor-1, \yEncryptor-3) {$m_2=F_2(m)$}; 
\node[align=center, purple] at (\xEncryptor+1, \yEncryptor-3) {$m_3=F_3(m)$}; 
\node[align=center, amaranth] at (\xEncryptor+1.7, \yEncryptor-2.3) {$m_4=F_4(m)$}; 
}





\onslide<2->
{
\draw[-triangle 60, black, thick] (\xEncryptor+1, \yEncryptor) to (\xMiddle, \yEmployeeProfile-0.2);
\draw[-triangle 60, black, thick] (\xEncryptor+1, \yEncryptor) to (\xMiddle,\yOfficeWorker);
\draw[-triangle 60, black, thick] (\xEncryptor+1, \yEncryptor) to (\xMiddle,\yTelephoneOperator);
\draw[-triangle 60, black, thick] (\xEncryptor+1, \yEncryptor) to (\xMiddle,\yManWithSunglasses);

\node[align=center, black] at (\xEncryptor ,\yEncryptor-1.2) {$C =\Enc (\pk, m)$}; 

\node[align=center,yellow] at (\xMiddle +1.8 ,\yEmployeeProfile-0.2) {$F_1(m)=\Dec(\sk_{F_1}, C)$}; 
\node[align=center,blue] at (\xMiddle +1.8,\yOfficeWorker) {$F_2(m)=\Dec(\sk_{F_2}, C)$}; 
\node[align=center,purple] at (\xMiddle +1.8,\yTelephoneOperator) {$F_3(m)=\Dec(\sk_{F_3}, C)$}; 
\node[align=center, amaranth] at (\xMiddle+1.8 ,\yManWithSunglasses) {$F_4(m)=\Dec(\sk_{F_4}, C)$}; 
\node[align=center,blue] at (\xMiddle +1.8,\yManWithSunglasses-0.4) {$F_2(m)=\Dec(\sk_{F_2}, C)$}; 

}

\end{tikzpicture}
\end{figure}
\end{frame}


\note{
This is exactly what functional encryption aims to do, only what I here called 'a part of the information on m' is a function of $m$, so a secret key will be associated to a given function. And given an encryption of $m$, and a secret key associated to a function $F$, decryption with this secret key will recover $F(m)$.
Applications:
\begin{itemize}
\item Statistical analysis on medical data
\item Spam filtering
\end{itemize}
}

\begin{frame}{Functional Encryption \cite{TCC:BonSahWat11}}
\begin{figure}
    \centering
\begin{tikzpicture}
    % Variables
% Alice x,y coordinates; width and height
\def\xalice{0}
\def\yalice{7}
\def\walice{1.5}
\def\halice{0.6}
\def\topalice{\yalice+\halice}

% bob x,y coordinates; width and height
\def\xbob{8}
\def\ybob{7}
\def\wbob{1.5}
\def\hbob{.6}
\def\topbob{\ybob+\hbob}

% authority x,y coordinates; width and height
\def\xauth{4}
\def\yauth{7}
\def\wauth{1.5}
\def\hauth{.6}
\def\topauth{\yauth+\hauth}

 \draw[rounded corners=5pt, mLightBrown, fill=mLightBrown]
  (\xbob,\ybob) rectangle ++(\wbob,\hbob);
  \node[align=left] at  (\xbob + \wbob/2 ,\ybob+\hbob/2) 
	{Bob};
\node[align=left, mLightBrown] at  (\xbob + \wbob/2 ,\ybob-\hbob/2) 
	{Function $F$};
	
\draw[rounded corners=5pt, mDarkBrown, fill=mDarkBrown]
  (\xauth,\yauth) rectangle ++(\wauth,\hauth);
  \node[align=left, black!2] at  (\xauth + \wauth/2 ,\yauth+\hauth/2) 
	{Auth.};
\node[align=left, mDarkBrown] at  (\xauth + \wauth/2 ,\yauth-\hauth/2) 
	{\textsf{Setup}};	
	\draw[-triangle 60, mDarkBrown] (\xauth + \wauth/2,\yauth-0.5) -- ++(0,-0.4);
\node[align=left, mDarkBrown] at  (\xauth + \wauth/2 ,\yauth-1.1) 
	{$(\mpk, \msk)$};
  
 \draw[rounded corners=5pt]
  (\xalice,\yalice) rectangle ++(\walice,\halice);
\node[align=left] at  (\xalice + \walice/2 ,\yalice+\halice/2) 
	{Alice};
\node[align=left] at  (\xalice + \walice/2 ,\yalice-\halice/2) 
	{m};	
	
\onslide<2>{
\draw[-triangle 60, mDarkBrown] (\xauth-0.5,\yauth-1.3) -- ++(-1,-0.25) 
	node [midway,above] {$\mpk$};
}

\onslide<2->{
\node[align=left] at  (\xalice + \walice/2 ,\yalice-1.6) 
	{$C =$ \textsf{Enc} $($\textcolor{mDarkBrown}{mpk}$, m)$};
}

\onslide<3>{
\draw[-triangle 60, mLightBrown] (\xbob-0.5,\ybob-0.75) -- ++(-1,-0.75) 
	node [midway,above] {$F$};
}
\onslide<3->{
\node[align=left,mDarkBrown] at  (\xauth + \wauth/2 ,\yauth-1.7) 
	{$\skf =$\textsf{KeyDer}$(\msk, $\textcolor{mLightBrown}{F}$)$};
}
\onslide<3>{	
\draw[-triangle 60, mDarkBrown] (\xauth+2.6,\yauth-1.7) -- ++(1,0) 
	node [midway,below] {$\skf$};
}

\onslide<3->{
\node[align=left, mLightBrown] at  (\xbob + \wbob/2 ,\ybob-1.7) 
	{$\skf$};
}


\onslide<4>{
\draw[-] (\xalice+\walice/2,\yalice - 2) -- ++(0,-1);
\draw[-triangle 60] (\xalice+\walice/2,\yalice - 3) -- ++(6.3,0) 
	node [midway,above] {$C$};
}

\onslide<4->{
\node[align=left, mLightBrown] at  (\xbob + \wbob/2 ,\ybob-3) 
	{$F(m) =$ \textsf{Dec}$(\skf, $\textcolor{mDarkTeal}{C}\textcolor{mLightBrown}{$)$}};
}

\onslide<4->{
\node[align=left] at  (4.5 ,\ybob-4.5) 
	{\large Bob \textbf{only learns} $F(m)$.};
}
\end{tikzpicture}
\end{figure}
\end{frame}

\begin{frame}{Associated Security Definition}
\begin{figure}
\begin{tikzpicture}

\def\ybotom{1}
\def\height{6}

% Adversary box
\draw[rounded corners=10pt]
  (0,\ybotom) rectangle ++(2,\height);

% Challenger box  
   \draw[rounded corners=10pt, fill=mDarkTeal, mDarkTeal]
  (3,\ybotom) rectangle ++(5.8,\height+0.5);
% FE scheme box  
    \draw[rounded corners=10pt, mLightBrown, fill=mDarkTeal]
  (5.5,\ybotom+1.6) rectangle ++(3.2,\height-1.6);
    
  
 \node[align=left] at (4,7.2) {\textbf{\textcolor{mLightBrown}{\textsf{Challenger}}}};
 \node[align=left] at (6.5,6.7) {\textcolor{mLightBrown}{\textsf{FE Scheme}}}; 
 \node[align=left] at (0.5,6.7) {\textsf{Adv}};

% Algorithms of the FE scheme 
  \node[align=left] at (6.3,6.2) {\textbf{\textcolor{black!2}{$\Setup$}}};
  \node[align=left] at (7.2,3) {\textbf{\textcolor{black!2}{$\Enc(\mpk, M_{b^*})$}}};
  
 % Arrows between FE and Challenger
 \draw[-triangle 60, white] (5.7,6.2) -- (5,6.2);
 \draw[-triangle 60, black!2] (5.7,3) -- (5,3);
 
 % Otputs of FE scheme
  \node[align=left] at (4.2,6.2) {\textbf{\textcolor{black!2}{$\mpk, \msk$}}}; 
  \node[align=left] at (4.2,3) {\textbf{\textcolor{black!2}{$C^*$}}}; 
  
  % Arrows between Challenger and Adv
   \draw[-triangle 60, mDarkTeal] (3,5.5) -- (1.5,5.5) node [midway,above ] {$\mpk$};
   \draw[-triangle 60, mDarkTeal] (1.5,4) -- (3,4) node [midway,above ] {$M_0, M_1$};
   \draw[-triangle 60, mDarkTeal] (3,2.8) -- (1.5,2.8) node [midway,above ] {$C^*$};
     \draw[-triangle 60, mDarkTeal] (1.5,1.6) -- (3,1.6) node [midway,above ] {\textbf{\textcolor{mDarkTeal}{$b$}}}; 
  
   
  % Challenger steps
  \node[align=left] at (4.2,3.8) {\textbf{\textcolor{black!2}{$b^* \rand \bit$}}}; 
          
          
 
   \onslide<1>{ 
  
  % Final test 
    \draw[rounded corners=5pt, fill=mLightBrown, mLightBrown]
  (4,\ybotom+0.3) rectangle ++(3.8,0.7);
  \node[align=left] at (5.9,1.6) {\textbf{\textcolor{mDarkTeal}{$b \; = \; b^*$}}}; 
  
  }


\onslide<2->{
% Key derivation oracle
\draw[rounded corners=10pt, fill=mLightBrown, mLightBrown]
  (-2.5,\ybotom) rectangle ++(2,\height);
  
\node[align=left] at (-1.5,6.7) {\textsf{KeyDer}}; 
  \node[align=left] at (-1.5,6.4) {\textsf{Oracle}}; 
   % Arrows between Adv and FE Scheme
   \draw[-triangle 60, mDarkTeal] (1,5) -- (-1,5) node [midway,above ] {$F_1, F_2 \dots$};
   \draw[-triangle 60, mDarkTeal] (-1,4.4) -- (1,4.4) node [midway,above ] {$\textsf{sk}_{F_1}, \textsf{sk}_{F_2} \dots$};
   
   \draw[-triangle 60, mDarkTeal] (1,2.5) -- (-1,2.5) node [midway,above ] {$F_q, F_{q+1} \dots$};
   \draw[-triangle 60, mDarkTeal] (-1,1.9) -- (1,1.9) node [midway,above ] {$\textsf{sk}_{F_q}, \textsf{sk}_{F_{q+1}} \dots$};   
}


  
   \onslide<2>{ 
  
  % Final test 
    \draw[rounded corners=5pt, fill=mLightBrown, mLightBrown]
  (4,\ybotom+0.1) rectangle ++(3.8,1.4);
  \node[align=left] at (5.9,\ybotom+1.1) {\textbf{\textcolor{mDarkTeal}{$\forall i, \quad F_i(M_0) = F_i(M_1)$}}};
    \node[align=left] at (5.9,\ybotom+0.7) {\textbf{\textcolor{mDarkTeal}{and}}};
  \node[align=left] at (5.9,\ybotom+0.3) {\textbf{\textcolor{mDarkTeal}{$b \; = \; b^*$}}}; 
  
  }  
  
\end{tikzpicture}
\end{figure}
\end{frame}

\note{
Indistinguishability (IND)

Adversary given access to $(\mathsf{sk}_{F_1}, \mathsf{sk}_{F_2}, \dots,  \mathsf{sk}_{F_q})$, cannot distinguish between $\Enc(\mpk, M_0)$ and $\Enc(\mpk, M_1)$ where $F_i(M_0) = F_i(M_1)$ for all i.

}

\begin{frame}{Limits of General Functional Encryption}
\begin{center}
\only<1>{
Constructions of FE for \textcolor{amaranth}{\textbf{general functions}} exist, but are  \textcolor{amaranth}{\textbf{not practical}}\\ 
\small{\cite{CCS:SahSey10,C:GorVaiWee12,C:GKPVZ13,STOC:GKPVZ13,C:ABSV15,C:Waters15,AC:BGJS16,TCC:GGHZ16}
}}
\only<2->{
Constructions of FE for {\textbf{general functions}} exists, but are {\textbf{not practical}}\\
\small{\cite{CCS:SahSey10,C:GorVaiWee12,C:GKPVZ13,STOC:GKPVZ13,C:ABSV15,C:Waters15,AC:BGJS16,TCC:GGHZ16}
}}
\normalsize
\vspace{.5cm}

\onslide<2->{
$\Rightarrow$ \textbf{Linear Functions:} \textcolor{darkjunglegreen}{\textbf{simple}} with \textcolor{darkjunglegreen}{\textbf{many applications}}
}

\end{center}
\onslide<3->{
\begin{itemize}
\item Understand general FE
\onslide<3->{\item Statistical analysis on encrypted data}
\onslide<3->{\item Evaluation of polynomials over encrypted data
\hfill \small{\cite{EC:KatSahWat08}}
\normalsize} 
\onslide<3->{\item Constructing trace-and-revoke systems
\hfill \small{\cite{CCS:ABPSY17}}
\normalsize} 
\onslide<3->{\item etc.}
\end{itemize}
}
\end{frame}

%%%%%%%%%%%%%%%%%%%%%%%%%%%%%%%%%%%%%%%%%%%%%%%%%%%%%%%%%%%%%%%%%%%%%%%%%%%%%%%
\subsection{The Inner Product Functionality}

\begin{frame}{The inner product functionality}
\begin{figure}
    \centering
\begin{tikzpicture}
    % Variables
% Alice x,y coordinates; width and height
\def\xalice{0}
\def\yalice{5}
\def\walice{1.5}
\def\halice{0.6}
\def\topalice{\yalice+\halice}

% bob x,y coordinates; width and height
\def\xbob{8}
\def\ybob{5}
\def\wbob{1.5}
\def\hbob{.6}
\def\topbob{\ybob+\hbob}

% authority x,y coordinates; width and height
\def\xauth{4}
\def\yauth{7}
\def\wauth{1.5}
\def\hauth{.6}
\def\topauth{\yauth+\hauth}

 \draw[rounded corners=5pt, mLightBrown, fill=mLightBrown]
  (\xbob,\ybob) rectangle ++(\wbob,\hbob);
  \node[align=left] at  (\xbob + \wbob/2 ,\ybob+\hbob/2) 
	{Bob};
\node[align=left, mLightBrown] at  (\xbob + \wbob/2 ,\ybob-\hbob/2) 
	{$\vec{x}, \skx$};
	
\draw[rounded corners=5pt, mDarkBrown, fill=mDarkBrown]
  (\xauth,\yauth) rectangle ++(\wauth,\hauth);
  \node[align=left, black!2] at  (\xauth + \wauth/2 ,\yauth+\hauth/2) 
	{Auth.};
\node[align=left, mDarkBrown] at  (\xauth + \wauth/2 ,\yauth-\hauth/2) 
	{\textsf{Setup}};	
	\draw[-triangle 60, mDarkBrown] (\xauth + \wauth/2,\yauth-0.5) -- ++(0,-0.4);
\node[align=left, mDarkBrown] at  (\xauth + \wauth/2 ,\yauth-1.1) 
	{$(\mpk, \msk)$};
  
 \draw[rounded corners=5pt]
  (\xalice,\yalice) rectangle ++(\walice,\halice);
\node[align=left] at  (\xalice + \walice/2 ,\yalice+\halice/2) 
	{Alice};
\node[align=left] at  (\xalice + \walice/2 ,\yalice-\halice/2) 
	{$\vec{y}$};	
	

\node[align=left] at  (\xalice + \walice/2 ,\yalice-0.8) 
	{$C =$ \textsf{Enc} $($\textcolor{mDarkBrown}{mpk}$, \vec{y})$};


\draw[-triangle 60] (\xauth-1.5,\yalice - 0.8) -- ++(4,0) 
	node [midway,above] {$C$};

\node[align=left, mLightBrown] at  (\xbob + \wbob/2 ,\ybob-0.8) 
	{$\langle \vec{x}, \vec{y} \rangle =$ \textsf{Dec}$(\skx, $\textcolor{mDarkTeal}{C}\textcolor{mLightBrown}{$)$}};

\node[align=left] at  (4.5 ,\ybob-2.5) 
	{\large $F_x : \mathcal{R}^{\ell} \mapsto \mathcal{R}$\\
	\hspace{.65cm} $y \quad \mapsto \langle \vec{x}, \vec{y} \rangle$};

\end{tikzpicture}
\end{figure}
\end{frame}

%%%%%%%%%%%%%%%%%%%%%%%%%%%%%%%%%%%%%%%%%%%%%%%%
\begin{frame}{Previous work}
\begin{figure}[H]
\begin{tikzpicture}
% Variables
% Context x,y coordinates; width and height
\def\xtime{0}
\def\ytime{0}
\def\wtime{9}
\def\htime{0.5}
\def\toptime{\ytime+\htime}

% width of an event in time
\def\wevent{0.05}

% PKC 2015 x,y coordinates; width and height
\def\pkcfifteen{0}

% Crypto 2016 x coordinate
\def\crypto16{2.3}

% iacr 2016 x coordinate
\def\iacr{4.3}

% pkc 2017 x coordinate
\def\pkcseventeen{6.1}

% asiacrypt 2018 x coordinate
\def\asiacrypt{\wtime - \wevent}

% Timeline
\draw[fill=mLightBrown, mLightBrown]
	(\xtime,\ytime) rectangle ++(\wtime,\htime);

% pkc2015
 \draw[fill=mDarkTeal, mDarkTeal] (\pkcfifteen,\ytime) rectangle ++(\wevent,\htime);
 \node[align=center] (pkc15) at  (\pkcfifteen,\ytime + 1) 
	{PKC 2015};
  \node[align=center] (pkc15) at  (\pkcfifteen,\ytime - 1) 
	{\small{\cite{PKC:ABDP15}}\\
    \scriptsize First IPFE schemes,\\
	\scriptsize from $\LWE$ and $\DDH$,\\
    \scriptsize only selectively secure.};
 
\onslide<1->{
% crypto 16
\draw[fill=mDarkTeal, mDarkTeal] (\crypto16,\ytime) rectangle ++(\wevent,\htime);
 \node[align=center] (crypto16) at  (\crypto16,\ytime + 1) 
	{Crypto 2016};
  \node[align=center] (crypto16) at (\crypto16,\ytime - 1) 
	{\small{\cite{C:AgrLibSte16}}\\
    \scriptsize Full security,\\
	\scriptsize from $\LWE$,\\
    \scriptsize $\DDH$ and $\DCR$.};    
}

\onslide<1->{
% iacr 16
\draw[fill=mDarkTeal, mDarkTeal] (\iacr,\ytime) rectangle ++(\wevent,\htime);
 \node[align=center] (iacr) at  (\iacr,\ytime + 1) 
	{2016}; 
   \node[align=center] (iacr) at (\iacr,\ytime - 1) 
	{\small{\cite{EPRINT:ABDP16}}\\
	\\
	\\};    	
	
}

\onslide<1->{
% pkc 17
\draw[fill=mDarkTeal, mDarkTeal] (\pkcseventeen,\ytime) rectangle ++(\wevent,\htime);
 \node[align=center] (pkc17) at  (\pkcseventeen,\ytime + 1) 
	{PKC 2017};
  \node[align=center] (pkc17) at (\pkcseventeen,\ytime - 1) 
	{\small{\cite{PKC:BenBouLip17}}\\
    \scriptsize Generic\\
	\scriptsize constructions\\
    \scriptsize from HPS.};
}

\onslide<1->{

\draw [decorate,decoration={brace,amplitude=4pt},xshift=0cm,yshift=0.5pt]
       (\pkcfifteen, \ytime+1.5) -- (\pkcseventeen, \ytime + 1.5) 
       node [midway,above,yshift=1.2cm] {\small Schemes mod $p$ do not recover}
       node [midway,above,yshift=0.7cm] {\small large inner products} 
       node [midway,above,yshift=0.2cm] {\small or are inefficient.}; 
}

\onslide<2->{
% asiacrypt 2018
\draw[fill=mDarkTeal, mDarkTeal] (\asiacrypt,\ytime) rectangle ++(\wevent,\htime);
 \node[align=center] (asiacrypt) at  (\asiacrypt,\ytime + 1) 
	{Asiacrypt 2018};
  \node[align=center] (asiacrypt) at (\asiacrypt,\ytime - 1.3) 
	{\scalebox{0.9}{\cite{AC:CasLagTuc18}}\\
    \scalebox{0.8}{IPFE mod $p$}\\
    \scalebox{0.8}{adaptive security}\\
	\scalebox{0.8}{no restriction on size}\\
    \scalebox{0.8}{and efficient!}};
}

\end{tikzpicture}
\end{figure}
\end{frame}



\note{
IPFE tends to use underlying homomorphic encryption schemes due to linearity property of inner product.

A part from the LWE based schemes of [ALS16], all IPFE schemes to date encrypt the message in the exponent of a group $G$ so as to benefit of the linear operations one can perform in the exponents by multiplying group elements.

Problem: decryption is only possible if the resulting inner product is small.

Note: ALS resolved this issue with an LWE based scheme, but their scheme for computing inner products modulo a prime $p$ is very inefficient (especially for vectors with many coordinates).

This work: We thereby present the first IPFE schemes which are both efficient and recover $\langle x, y \rangle \mod p$ whatever its size.
}


%%%%%%%%%%%%%%%%%%%%%%%%%%%%%%%%%%%%%%%%%%%%%%%%%%%%%%%%%%%%%%%%%%%%%%%%%%%%%%%
\section{IPFE modulo \texorpdfstring{$p$}{p} from [CL15]}

\begin{frame}{IPFE scheme mod $p$ from $\HSM$ \small(simplified)}
\begin{figure}[H]
\begin{tikzpicture}

% Setup x,y coordinates; width and height
\def\xsetup{-1}
\def\ysetup{9}
\def\wsetup{11}
\def\hsetup{1}
\def\topsetup{\ysetup+\hsetup}

% Enc x,y coordinates; width and height
\def\wenc{11}
\def\henc{2}
\def\xenc{-1}
\def\yenc{\ysetup - \henc - 0.2}
\def\topenc{\yenc+\henc}

% Key derivation x,y coordinates; width and height
\def\wKd{11}
\def\hKd{1.1}
\def\xKd{-1}
\def\yKd{\yenc-\hKd-0.2}
\def\topKd{\yKd+\hKd}

% Decryption x,y coordinates; width and height
\def\hdec{2}
\def\wdec{11}
\def\xdec{-1}
\def\ydec{\yKd-\hdec-0.2}
\def\topdec{\ydec+\hdec}

% Small spacing between text
\def\swspace{0.4}

% Normal spacing between text
\def\nwspace{0.5}

% Normal spacing top and first text
\def\ntopspace{0.3}



% Orange boxes:
\onslide<1>{
% Setup 
 \draw[rounded corners=5pt, mLightBrown, fill=mLightBrown]
  (\xsetup,\ysetup) rectangle ++(\wsetup,\hsetup);
}

\onslide<2>{
% Encryption
 \draw[rounded corners=5pt, mLightBrown, fill=mLightBrown]
  (\xenc,\yenc) rectangle ++(\wenc,\henc);
}
%\onslide<3>{
%% Encryption
% \draw[rounded corners=5pt, mLightBrown, fill=mLightBrown]
%  (\xenc,\yenc) rectangle ++(\wenc,\henc);
%}

\onslide<3>{
% Key derivation
 \draw[rounded corners=5pt, mLightBrown, fill=mLightBrown]
  (\xKd,\yKd) rectangle ++(\wKd,\hKd);
}


% Setup
\onslide<1->{
\node[align=left] at  (\xsetup + 0.7 ,\topsetup - \ntopspace) {\underline{\textbf{Setup}}};

\node[align=left] at  (\xsetup + 3.2 ,\topsetup - \ntopspace) 
	{Sample $\vec{t}=(t_1, \dots,
    t_{\ell})$};
\node[align=left] at  (\xsetup + 8 ,\topsetup - \ntopspace)  
	{compute $\quad h_i = g_p^{t_i}\;$ for $i=1,\dots , \ell$};
\node[align=left] at  (\xsetup + \wsetup/2 ,\topsetup - \ntopspace - \nwspace)  
	{$\msk = \vec{t}\;$ and $\;\mpk = (h_1,\dots, h_{\ell})$}; 
}

% Now on fourth frame we present the Encrypt algorithm
% Setup goes to normal color


\onslide<2->{
% Setup 
 \draw[rounded corners=5pt]
  (\xsetup,\ysetup) rectangle ++(\wsetup,\hsetup);

\node[align=left] at  (\xenc + 0.45,\topenc - 0.25) {\underline{\textbf{Enc}}};
\node[align=left] at  (\xenc + 3.8,\topenc - 0.25) 
	{
    \textbf{Plaintext:} $\vec{y} = (y_1,\dots, y_{\ell}) \in (\Zp)^{\ell}$};
\node[align=left] at  (\xenc + 5.4,\topenc - 1.2) 
	{
    \textbf{Sample randomness} $r$\\
    \textbf{Ciphertext:}\\
    \hspace{2cm}$\vec{C} = (C_0=g_p^r, C_1=f^{y_1}\cdot h_1^r, \dots , C_{\ell}=f^{y_{\ell}}\cdot h_{\ell}^r)$};
}



% Fifth frame: present the KeyDer algorithm
% Setup, Encrypt: normal color

\onslide<3->{
% Encryption white
 \draw[rounded corners=5pt]
  (\xenc,\yenc) rectangle ++(\wenc,\henc);

\node[align=left] at  (\xKd + 0.8 ,\topKd - 0.3) {\underline{\textbf{KeyDer}}};
\node[align=left] at  (\xKd + 4.2 ,\topKd -0.5)
	{
	\textbf{Input:} $\vec{x}= (x_1,\dots, x_{\ell}) \in (\Zp)^{\ell}$\\
	\textbf{Output key:} $\skx = \langle\vec{t},\vec{x} \rangle$
    };
}


% 10+ frame: present the Decrypt algorithm
% Setup, Encrypt, KeyDer: normal color

\onslide<4->{
% Key derivation
 \draw[rounded corners=5pt]
  (\xKd,\yKd) rectangle ++(\wKd,\hKd);
  
% Decryption
 \draw[rounded corners=5pt, mLightBrown, fill=mLightBrown]
  (\xdec,\ydec) rectangle ++(\wdec,\hdec);

\node[align=left] at  (\xdec + 0.45 ,\topdec - 0.35) {\underline{\textbf{Dec}}};



			\node[align=left] (input) at  
			(\xdec + 2.3,\topdec - \ntopspace) 
			{\textbf{From $\vec{C}, \vec{x}$ and $\skx$ :}};


\only<5->{
			\node[align=left] (input) at  
			(\xdec + 4.5 ,\topdec - \ntopspace) 
			{$\;\prod_{i=1}^\ell C_i^{{x}_i}$};
}

\only<6>{
			\node[align=left] (input) at  
			(\xdec + 6.32 ,\topdec - \ntopspace-0.05) 
			{$=\prod%_{i=1}^\ell 
            (f^{\;y_i} \cdot h_i^{r})^{x_i}$};
}  

\only<7>{
			\node[align=left] (input) at  
			(\xdec + 6.57 ,\topdec - \ntopspace) 
			{$=f^{\;\sum y_i x_i}\cdot g_p^{r\cdot \sum t_i x_i}$};
}

\only<8->{
			\node[align=left] (input) at  
			(\xdec + 6.4 ,\topdec - \ntopspace) 
			{$=f^{\;\langle \vec{y}, \vec{x} \rangle}
            \cdot g_p^{r\cdot \langle \vec{t}, \vec{x} \rangle}$};
}

\only<9->{
			\node[align=left] (input) at  
            (\xdec + 8.9 ,\topdec - \ntopspace) 
			{\textbf{and} $\quad C_0^{\skx} =$};

			\node[align=left] (input) at  
            (\xdec + 10.4 ,\topdec - \ntopspace)
			{$g_p^{r \cdot\langle \vec{t},\vec{x} \rangle}$};
}

\only<10->{
			\node[align=left] at  (\xdec + 1.8 ,\topdec - \ntopspace - 0.7) 
            {\textbf{Such that: }};
            
            \node[align=left] at (\xdec + 5, \topdec - \ntopspace - 1) 
            {$\prod_{i=1}^\ell C_i^{{x}_i} / C_0^{\skx} 
            \; = f^{\langle \vec{x}, \vec{y} \rangle}$};

\draw[-triangle 60, mDarkBrown] (\xdec + 6.8, \topdec - \ntopspace - 1) -- ++(1,0) 
	node [midway,above] {$\DL$};

%            \node[align=left] at (\xdec + 5.93, \topdec - \ntopspace - 1) 
%            {$\DL(\prod_{i=1}^\ell C_i^{{x}_i} / C_0^{\skx})
%            = \langle \vec{x}, \vec{y} \rangle$ mod $p$};
}

            \node[align=left] at (\xdec + 9, \topdec - \ntopspace - 1) 
            {$\langle \vec{x}, \vec{y} \rangle$ mod $p$};

}
\end{tikzpicture}
\end{figure}
\end{frame}



{\setbeamercolor{palette primary}{fg=black!2, bg=mDarkTeal}
\begin{frame}[standout]{Security}
  This scheme is \textcolor{mLightBrown}{secure} under the \textcolor{mLightBrown}{$\HSM$} assumption.
\end{frame}
}

\begin{frame}{Proof technique}
\begin{figure}
\begin{tikzpicture}
\onslide<1>{
\node[align=left] at  (0,0) 
	{$\vec{C} = (C_0=g_p^r, C_1=f^{y_{\textcolor{amaranth}{b^*},1}}\cdot h_1^r, \dots , C_{\ell}=f^{y_{\textcolor{amaranth}{b^*},\ell}}\cdot h_{\ell}^r)$};
	}

\onslide<2>{
\node[align=left] at  (0,0) 
	{$\vec{C} = (C_0=g_p^r, C_1=f^{y_{\textcolor{amaranth}{b^*},1}}\cdot \textcolor{mLightBrown}{C_0^{t_1}}, \dots , C_{\ell}=f^{y_{\textcolor{amaranth}{b^*},\ell}}\cdot \textcolor{mLightBrown}{C_0^{t_{\ell}}})$};
	}
	
\onslide<3->{
\node[align=left] at  (0,0) 
	{$\vec{C} = (C_0=g_p^r\textcolor{mLightBrown}{f^u}, C_1=f^{y_{\textcolor{amaranth}{b^*},1}}\cdot C_0^{t_1}, \dots , C_{\ell}=f^{y_{\textcolor{amaranth}{b^*},\ell}}\cdot C_0^{t_{\ell}})$};
	}

\end{tikzpicture}
\end{figure}
\onslide<1->{
\begin{itemize}
\item \textcolor{mLightBrown}{Game 0} original security game
\onslide<2->{\item \textcolor{mLightBrown}{Game 1} use \textbf{secret key} to compute challenge ciphertext}% \cite{EC:CraSho02}
\onslide<3->{\item \textcolor{mLightBrown}{Game 2} indistinguishable from Game 1 under the $\HSM$ assumption.}
\end{itemize}
}

\onslide<4->{
In Game $2$, from $\A$'s view \textcolor{amaranth}{$b^*$ is \textbf{statistically hidden}}, given
\begin{itemize}
\item the public key
\item the challenge ciphertext
\item key derivation queries
\end{itemize}
}
\end{frame}



\section{Enhancing Security}
\begin{frame}{Active adversaries}
What can an \textbf{\textcolor{red}{active adversary}} do?
\begin{itemize}
\item Everything a passive adversary can do (observe)
\item Deviate from the protocol (act)
\item[$\Rightarrow$] Request the decryption of ciphertexts of its choice
\begin{itemize}
\item Ciphertexts may not have been obtained from encryption protocol
\item Malformed ciphertexts could leak information on secret keys!
\end{itemize}
\end{itemize}
\end{frame}

\begin{frame}{Protecting against active adversaries}
\begin{center}
Passive attacks $\Rightarrow$ \textcolor{red}{Confidentiality}
\end{center}
%\vspace{2cm}
\begin{center}
Active attacks $\Rightarrow$ \textcolor{red}{Ciphertext Integrity}

$\rightsquigarrow$ ensures adversary has not deviated from protocol

(no more info than CPA)
\end{center}
\end{frame}

\begin{frame}{CCA security from the Cramer Shoup paradigm [CS02]}
Cramer and Shoup introduce \textbf{Hash Proof Systems} (HPS):
\begin{itemize}
\item[-] Useful tool to abstract the properties of a cryptosystem
\item[-] [CS02] generic construction from HPS to CCA secure encryption
\item[-] Specific properties of HPS ensure \textbf{confidentiality} and \textbf{integrity}
\item[-] [BBL17] generic construction from homomorphic HPS to CCA secure IPFE
\end{itemize}
\end{frame}

\begin{frame}{HPS in the CL framework}
\begin{itemize}
\item We build homomorphic HPS from assumptions in CL framework
\item We improve [BBL17] by tightening security proof
\item We build CCA secure PKE and IPFE from CL framework
\end{itemize}
\end{frame}


\section{Threshold Signatures}

\begin{frame}{Digital Signatures}
\textcolor{red}{Goal:} \textbf{Authenticity}, \textbf{Integrity} and \textbf{Non Repudiation} of e-documents

\textcolor{red}{Algorithms:} 
\begin{itemize}
\item \textcolor{mLightBrown}{\textbf{$\Kg$($1^\lambda$)}} $\rightarrow$ secret signing key $\sk$ and public verification key $\vk$
\item \textcolor{mLightBrown}{\textbf{$\Sign$($\sk,m$)}} $\rightarrow$ signature $\sigma$ for $m$ under signing key $\sk$
\item \textcolor{mLightBrown}{\textbf{$\Verif$($\vk,\sigma, m$)}} $\rightarrow$ \textsf{accept}/\textsf{reject}
\end{itemize}
\textcolor{red}{Security (informal):} cannot produce valid signature for $m$ under $\vk$ without knowing $\sk$, even given many $\sigma'=\Sign(\sk,m')$ for $m'\neq m$.
\end{frame}

\begin{frame}{Threshold Cryptography -- Signatures}

\textcolor{red}{$(n,t)$ secret sharing scheme} 
\begin{itemize}
\item Share secret $S$ among $n$ users, require $t$ users for reconstruction
\end{itemize}
\textcolor{red}{$(n,t)$ threshold signature scheme} 
\begin{itemize}
\item Share private \textbf{signing key} $\sk$ among $n$ users, require $t$ users for \textbf{signing}
%\item Participant $P_i$ gets share $sk_i$
\end{itemize}
\end{frame}

\begin{frame}{ECDSA}
\textcolor{mLightBrown}{ECDSA} Digital signature that uses elliptic curve cryptography

\textcolor{red}{Threshold ECDSA} in high demand:
\begin{itemize}
\item secure cryptocurrency wallets $\rightsquigarrow$ protect key against theft
\item cryptocurrency custody solutions $\rightsquigarrow$ split key between bank/customer/3rd party trustee
\end{itemize}
\textcolor{red}{Difficult:} share $k = k_1 + k_2 + \dots + k_n \mod q$, \\
\hspace{1.3cm} need to compute $k^{-1} \mod q$ to sign.

\textcolor{darkjunglegreen}{[Lin17]} efficient solution for 2-Party $(n=t=2)$\\
\hspace{1cm} \textcolor{red}{Problem:} interactive, non standard, assumption.

\end{frame}

\begin{frame}{Tight security for two-Party ECDSA from HPS}
\begin{itemize}
\item Generic construction from homomorphic HPS to 2-Party ECDSA
\item Tight security proof
\item Instantiation in CL framework
\begin{itemize}
\item \textbf{Timings:} comparable to [Lin17]
\item \textbf{Communication cost:} significantly less than [Lin17]
\end{itemize}
\end{itemize}
\end{frame}


%%%%%%%%%%%%%%%%%%%%%%%%%%%%%%%%%%%%%%%%%%%%%%%%%%%%%%
\begin{frame}{Last slide!}
\begin{block}{Conclusion}
\begin{itemize}
\item Most efficient CPA-secure IPFE schemes to date
\item First IPFE mod a prime that recover the result whatever its size
\item Homomorphic Hash Proof Systems from CL framework
\item Chosen Ciphertext Attack Secure IPFE schemes 
\item 2-Party ECDSA from Homomorphic Hash Proof Systems
\end{itemize}
\end{block}
\begin{block}{Ongoing work}
\begin{itemize}
\item Full Threshold ECDSA
\end{itemize}
\end{block}
\end{frame}

\begin{frame}[standout]
  Questions?
\end{frame}


\begin{frame}
%
% ---- Bibliography ----
%
\bibliographystyle{../alpha-short}
\bibliography{../cryptobib/abbrev3,../cryptobib/crypto}
\end{frame}
\end{document}

