\begin{frame}{Functional Encryption}
\begin{figure}
    \centering
\begin{tikzpicture}
    % Variables
% Alice x,y coordinates; width and height
\def\xEmployees{8}

\def\yEmployeeProfile{8}
\def\yOfficeWorker{6}
\def\yTelephoneOperator{4}
\def\yManWithSunglasses{2}


% Encryptor x,y coordinates; width and height
\def\xEncryptor{0}
\def\yEncryptor{5}

% Other
\def\xMiddle{4}


\node[inner sep=0pt] at (\xEmployees ,\yEmployeeProfile)
 {\includegraphics[width=0.1\textwidth]{img/employee-profile.png}};
 
\node[inner sep=0pt] at (\xEmployees ,\yOfficeWorker)
 {\includegraphics[width=0.1\textwidth]{img/office-worker-outline.png}};

\node[inner sep=0pt] at (\xEmployees ,\yTelephoneOperator)
 {\includegraphics[width=0.1\textwidth]{img/telephone-operator.png}};
 
\node[inner sep=0pt] at (\xEmployees ,\yManWithSunglasses)
 {\includegraphics[width=0.1\textwidth]{img/man-with-sunglasses.png}};

\node[inner sep=0pt] at (\xEncryptor ,\yEncryptor)
 {\includegraphics[width=0.1\textwidth]{img/document.png}};

\node[align=center, yellow] at (\xEmployees ,\yEmployeeProfile-0.8) {$\sk_{F_1}$}; 
\node[align=center, blue] at (\xEmployees ,\yOfficeWorker-0.8) {$\sk_{F_2}$}; 
\node[align=center, purple] at (\xEmployees ,\yTelephoneOperator-0.8) {$\sk_{F_3}$}; 
\node[align=center, amaranth] at (\xEmployees ,\yManWithSunglasses-0.8) {$\sk_{F_4}, \textcolor{blue}{\sk_{F_2}}$}; 
%\node[align=center, black] at (\xEncryptor ,\yEncryptor+1) {$\msk, \pk$}; 
\node[align=center, black] at (\xEncryptor ,\yEncryptor-0.8) {$m$}; 

\onslide<1>{
\draw[-triangle 60, yellow, thick] (\xEncryptor ,\yEncryptor-1) to (\xEncryptor-1.5, \yEncryptor-2);
\draw[-triangle 60, blue, thick] (\xEncryptor ,\yEncryptor-1) to (\xEncryptor-0.5, \yEncryptor-2.7);
\draw[-triangle 60, purple, thick] (\xEncryptor ,\yEncryptor-1) to (\xEncryptor+0.5, \yEncryptor-2.7);
\draw[-triangle 60, amaranth, thick] (\xEncryptor ,\yEncryptor-1) to (\xEncryptor+1.5, \yEncryptor-2);

\node[align=center, yellow] at (\xEncryptor-1.7, \yEncryptor-2.3) {$m_1=F_1(m)$}; 
\node[align=center, blue] at (\xEncryptor-1, \yEncryptor-3) {$m_2=F_2(m)$}; 
\node[align=center, purple] at (\xEncryptor+1, \yEncryptor-3) {$m_3=F_3(m)$}; 
\node[align=center, amaranth] at (\xEncryptor+1.7, \yEncryptor-2.3) {$m_4=F_4(m)$}; 
}





\onslide<2->
{
\draw[-triangle 60, black, thick] (\xEncryptor+1, \yEncryptor) to (\xMiddle, \yEmployeeProfile-0.2);
\draw[-triangle 60, black, thick] (\xEncryptor+1, \yEncryptor) to (\xMiddle,\yOfficeWorker);
\draw[-triangle 60, black, thick] (\xEncryptor+1, \yEncryptor) to (\xMiddle,\yTelephoneOperator);
\draw[-triangle 60, black, thick] (\xEncryptor+1, \yEncryptor) to (\xMiddle,\yManWithSunglasses);

\node[align=center, black] at (\xEncryptor ,\yEncryptor-1.2) {$C =\Enc (\pk, m)$}; 

\node[align=center,yellow] at (\xMiddle +1.8 ,\yEmployeeProfile-0.2) {$F_1(m)=\Dec(\sk_{F_1}, C)$}; 
\node[align=center,blue] at (\xMiddle +1.8,\yOfficeWorker) {$F_2(m)=\Dec(\sk_{F_2}, C)$}; 
\node[align=center,purple] at (\xMiddle +1.8,\yTelephoneOperator) {$F_3(m)=\Dec(\sk_{F_3}, C)$}; 
\node[align=center, amaranth] at (\xMiddle+1.8 ,\yManWithSunglasses) {$F_4(m)=\Dec(\sk_{F_4}, C)$}; 
\node[align=center,blue] at (\xMiddle +1.8,\yManWithSunglasses-0.4) {$F_2(m)=\Dec(\sk_{F_2}, C)$}; 

}

\end{tikzpicture}
\end{figure}
\end{frame}


\note{
This is exactly what functional encryption aims to do, only what I here called 'a part of the information on m' is a function of $m$, so a secret key will be associated to a given function. And given an encryption of $m$, and a secret key associated to a function $F$, decryption with this secret key will recover $F(m)$.
Applications:
\begin{itemize}
\item Statistical analysis on medical data
\item Spam filtering
\end{itemize}
}