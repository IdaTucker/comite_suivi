%%%%%%%%%%%%%%%%%%%%%%%%%%%%%%%%%%%%%%%%%%%%%%%%
\begin{frame}{Previous work}
\begin{figure}[H]
\begin{tikzpicture}
% Variables
% Context x,y coordinates; width and height
\def\xtime{0}
\def\ytime{0}
\def\wtime{9}
\def\htime{0.5}
\def\toptime{\ytime+\htime}

% width of an event in time
\def\wevent{0.05}

% PKC 2015 x,y coordinates; width and height
\def\pkcfifteen{0}

% Crypto 2016 x coordinate
\def\crypto16{2.3}

% iacr 2016 x coordinate
\def\iacr{4.3}

% pkc 2017 x coordinate
\def\pkcseventeen{6.1}

% asiacrypt 2018 x coordinate
\def\asiacrypt{\wtime - \wevent}

% Timeline
\draw[fill=mLightBrown, mLightBrown]
	(\xtime,\ytime) rectangle ++(\wtime,\htime);

% pkc2015
 \draw[fill=mDarkTeal, mDarkTeal] (\pkcfifteen,\ytime) rectangle ++(\wevent,\htime);
 \node[align=center] (pkc15) at  (\pkcfifteen,\ytime + 1) 
	{PKC 2015};
  \node[align=center] (pkc15) at  (\pkcfifteen,\ytime - 1) 
	{\small{\cite{PKC:ABDP15}}\\
    \scriptsize First IPFE schemes,\\
	\scriptsize from $\LWE$ and $\DDH$,\\
    \scriptsize only selectively secure.};
 
\onslide<1->{
% crypto 16
\draw[fill=mDarkTeal, mDarkTeal] (\crypto16,\ytime) rectangle ++(\wevent,\htime);
 \node[align=center] (crypto16) at  (\crypto16,\ytime + 1) 
	{Crypto 2016};
  \node[align=center] (crypto16) at (\crypto16,\ytime - 1) 
	{\small{\cite{C:AgrLibSte16}}\\
    \scriptsize Full security,\\
	\scriptsize from $\LWE$,\\
    \scriptsize $\DDH$ and $\DCR$.};    
}

\onslide<1->{
% iacr 16
\draw[fill=mDarkTeal, mDarkTeal] (\iacr,\ytime) rectangle ++(\wevent,\htime);
 \node[align=center] (iacr) at  (\iacr,\ytime + 1) 
	{2016}; 
   \node[align=center] (iacr) at (\iacr,\ytime - 1) 
	{\small{\cite{EPRINT:ABDP16}}\\
	\\
	\\};    	
	
}

\onslide<1->{
% pkc 17
\draw[fill=mDarkTeal, mDarkTeal] (\pkcseventeen,\ytime) rectangle ++(\wevent,\htime);
 \node[align=center] (pkc17) at  (\pkcseventeen,\ytime + 1) 
	{PKC 2017};
  \node[align=center] (pkc17) at (\pkcseventeen,\ytime - 1) 
	{\small{\cite{PKC:BenBouLip17}}\\
    \scriptsize Generic\\
	\scriptsize constructions\\
    \scriptsize from HPS.};
}

\onslide<1->{

\draw [decorate,decoration={brace,amplitude=4pt},xshift=0cm,yshift=0.5pt]
       (\pkcfifteen, \ytime+1.5) -- (\pkcseventeen, \ytime + 1.5) 
       node [midway,above,yshift=1.2cm] {\small Schemes mod $p$ do not recover}
       node [midway,above,yshift=0.7cm] {\small large inner products} 
       node [midway,above,yshift=0.2cm] {\small or are inefficient.}; 
}

\onslide<2->{
% asiacrypt 2018
\draw[fill=mDarkTeal, mDarkTeal] (\asiacrypt,\ytime) rectangle ++(\wevent,\htime);
 \node[align=center] (asiacrypt) at  (\asiacrypt,\ytime + 1) 
	{Asiacrypt 2018};
  \node[align=center] (asiacrypt) at (\asiacrypt,\ytime - 1.3) 
	{\scalebox{0.9}{\cite{AC:CasLagTuc18}}\\
    \scalebox{0.8}{IPFE mod $p$}\\
    \scalebox{0.8}{adaptive security}\\
	\scalebox{0.8}{no restriction on size}\\
    \scalebox{0.8}{and efficient!}};
}

\end{tikzpicture}
\end{figure}
\end{frame}



\note{
IPFE tends to use underlying homomorphic encryption schemes due to linearity property of inner product.

A part from the LWE based schemes of [ALS16], all IPFE schemes to date encrypt the message in the exponent of a group $G$ so as to benefit of the linear operations one can perform in the exponents by multiplying group elements.

Problem: decryption is only possible if the resulting inner product is small.

Note: ALS resolved this issue with an LWE based scheme, but their scheme for computing inner products modulo a prime $p$ is very inefficient (especially for vectors with many coordinates).

This work: We thereby present the first IPFE schemes which are both efficient and recover $\langle x, y \rangle \mod p$ whatever its size.
}
